%% start of file `template.tex'.
%% Copyright 2006-2012 Xavier Danaux (xdanaux@gmail.com).
%
% This work may be distributed and/or modified under the
% conditions of the LaTeX Project Public License version 1.3c,
% available at http://www.latex-project.org/lppl/.

\documentclass[11pt,a4paper,sans]{moderncv}

% moderncv themes
\moderncvstyle{casual}                        % style options are 'casual' (default), 'classic', 'oldstyle' and 'banking'
\moderncvcolor{grey}                          % color options 'blue' (default), 'orange', 'green', 'red', 'purple', 'grey' and 'black'
%\renewcommand{\familydefault}{\sfdefault}    % to set the default font; use '\sfdefault' for the default sans serif font, '\rmdefault' for the default roman one, or any tex font name
\nopagenumbers{}                             % uncomment to suppress automatic page numbering for CVs longer than one page
% adjust the page margins
\usepackage[scale=0.85]{geometry}
%\setlength{\hintscolumnwidth}{3cm}           % if you want to change the width of the column with the dates
\setlength{\makecvtitlenamewidth}{10cm}      % for the 'classic' style, if you want to force the width allocated to your name and avoid line breaks. be careful though, the length is normally calculated to avoid any overlap with your personal info; use this at your own typographical risks...

% personal data
\firstname{}
\familyname{Nathan Burgers}
\title{Full-Stack Software Engineer}                          % optional, remove / comment the line if not wanted
\address{65 Beachridge Drive}{14051 East Amherst}    % optional, remove / comment the line if not wanted
\mobile{+1~(716)~697~6060}                     % optional, remove / comment the line if not wanted
%\phone{+1~()~678~901}                      % optional, remove / comment the line if not wanted
\email{nateburgers@gmail.com}                          % optional, remove / comment the line if not wanted
%\homepage{nateb.me}                    % optional, remove / comment the line if not wanted
%\extrainfo{additional information}            % optional, remove / comment the line if not wanted
%\photo[64pt][0.4pt]{picture}                  % optional, remove / comment the line if not wanted; '64pt' is the height the picture must be resized to, 0.4pt is the thickness of the frame around it (put it to 0pt for no frame) and 'picture' is the name of the picture file
%\quote{Some quote}                            % optional, remove / comment the line if not wanted

%----------------------------------------------------------------------------------
%            content
%----------------------------------------------------------------------------------
\begin{document}
\makecvtitle

\section{Education}
\cventry{2012-Present}{Computer Science}{University at Buffalo}{}{\textit{3.7/4.0}}{}

\section{Professional Experience}
\cventry{2015-Present}{Systems Developer}{Bloomberg LP}{New York}{New York}
        {Work on a team of developers that create new developer tools and user experience technologies for the Bloomberg Professional Service. \newline
  \textbf{Detailed achievements:}
  \begin{itemize}
  \item Created and peer reviewed proposals for new user experience systems
  \item Introduced new features into large, existing codebases
  \item Created software libraries that enable trusted application configuration
  \end{itemize}}
        
\cventry{2011-2014}{iOS \& Full-Stack Web Developer}{Refulgent Software LLC.}{Amherst}{New York}
{Work on a team of developers that create a restaurant Point of Sale system for iOS. Design, develop, and maintain secure automated financial systems.
\newline
\textbf{Detailed achievements:}
\begin{itemize}%
\item Actively worked with a team on a large-scale Objective-C code-base
\item Created functional Objective-C libraries that provide: infinite data structures, implicit memoization, and parser generation
\item Automated software licensing over encrypted, timestamped channels
\item Developed API frontend
\item Designed and developed new iOS UI components
\end{itemize}}

\section{Open Source Work}
\cventry{2013-2015}{MLRTEMS}{Research under Professor Lukasz Ziarek}{University at Buffalo}{}
{
An extension of the MLton Standard-ML compiler for interfacing with the RTEMS real-time embedded systems executive, which aims to provide real-time guarantees to embedded functional programming.
}

\section{Extracurricular Activity}
\cvitem{2013} {Creator of the Lark Language, Top 10, Most Technically Challenging, and Best iOS App at MHacks \newline \url{github.com/nateburgers/LarkDemo}}
\cvitem{2013} {iOS Developer of Playper, Top 10 at HackMIT \newline \url{github.com/nateburgers/Playper}}
%\cvitem{2013} {Back-End Audio Developer of Theramixer, Second place at HackPrinceton \newline \url{github.com/buffalohackers/Theramixer}}
%\cvitem{2013} {WebRTC Developer of WebDrop, First Place at Hack Upstate \newline \url{github.com/buffalohackers/WebDrop}}
\cvitem{2013} {Google Summer of Code: Developed an automatic, ad-hoc build toolchain for the RTEMS real-time embedded systems executive.}
\cvitem{2012-Present} {Member of the University at Buffalo chapter of The Association for Computing Machinery}
\cvitem{2012-Present} {Volunteer for UBHacking, the University at Buffalo Hackathon}

\section{Skills}
\cvitem{Development}{Languages and DSLs, Back-End Server Systems, iOS}
\cvitem{Languages}{C++, C, Objective-C, Haskell, Ocaml, Standard-ML, Erlang (\& Familiar with Others)}

\end{document}
